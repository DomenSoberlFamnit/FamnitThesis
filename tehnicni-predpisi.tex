\chapter{TEHNIČNI PREDPISI}

Zaključna naloga mora biti napisana v pokončnem formatu A4 (strani bele barve). Po končni potrditvi mentorja, morebitnega somentorja, komisije (če je komisija imenovana, kar je odvisno od določil pravilnika o zaključni nalogi posameznega programa) in potrditve tehnične ustreznosti zaključne naloge, študent odda en izvod v PDF formatu v Referat za
študente. Študent posreduje končni potrjen izvod na e-naslov Referata.

Stran naj bo oblikovana tako, da so robovi široki 2,5 cm, glava besedila pa je 1,5 cm (od vrha strani). Če uporabljena programska oprema, ne omogoča točno takšne nastavitve robov in glave besedila, uporabite najbolj podobno.

Besedilo je praviloma v pisavi Times New Roman, velikosti 12 pt, z obojestransko
poravnavo ter razmikom med vrsticami 1,25. Če uporabljena programska oprema, ne omogoča točno takšne nastavitve pisave, uporabite najbolj podobno.

\emph{Opombe} pod črto (footnotes) se pišejo v velikosti 10 pt.

\section{ŠTEVILČENJE STRANI/LISTOV}

Naslovna stran (Priloga B, Priloga F) ni oštevilčena, čeprav velja kot prva stran. Začetne splošne strani so številčene z rimskimi števili od I dalje – in sicer od naslovne strani do zadnje strani kazal oz. seznama kratic.

Z arabskimi številkami začnemo s poglavjem 1 (npr. Uvod) in končamo s poglavjem Literatura. Številka strani je v desnem zgornjem kotu strani, v tekočem ali sprotnem naslovu.

\emph{Tekoči ali sprotni naslov} (running head ali pagina viva) naj bo velikosti 10 pt in oblikovan po vzorcu te strani.

V kolikor naslov zaključne naloge presega vrstico, slednjega v dogovoru z mentorjem in morebitnim somentorjem primerno okrajšajte z uporabo tropičja. Okrajšani naslov se zapisuje samo v tekočem ali sprotnem naslovu (running head ali pagina viva). Na platnici, naslovni strani in ključni dokumentacijski informaciji zapisujete celoten naslov zaključne
naloge

\section{IZDAJA POD LICENCAMI}

Če študent v soglasju z mentorjem odloči, da zaključno nalogo izda pod licencami, ki ponuja določen del pravic vsem (na primer Creative Commons ali GNU GPL), vključi stran z opisom uporabljenih licenc v uvodnem delu naloge, to je takoj za naslovno stranjo (prva notranja stran).

\section{ZAHVALA}

Za naslovno stranjo (prva notranja stran), ki je stran s podatki oz opisom uporabljenih licenc (če je zaključno delo izdano pod licencami), sledi zahvala. V njej se zahvalimo sodelujočim pri zaključni nalogi, osebam ali ustanovam, ki so nam pri delu pomagale ali so delo omogočile. Zahvalimo se lahko tudi mentorju in morebitnemu somentorju.

\section{KLJUČNA DOKUMENTACIJSKA INFORMACIJA}

Za zahvalo sledi stran s podatki o zaključni nalogi - ključna dokumentacijska informacija (KDI) in key document information — ključna dokumentacijska informacija v angleščini.

\section{KAZALA}

Kazala sledijo ključni dokumentacijski informaciji.
Pripraviti je treba vsa ustrezna kazala v naslednjem vrstnem redu:

\begin{itemize}
\item kazalo vsebine,
\item kazalo preglednic,
\item kazalo slik in grafikonov,
\item kazalo prilog,
\item seznam kratic.
\end{itemize}

\section{ŠTEVILČENJE POGLAVIJ}

Za številčenje poglavij uporabljamo arabska števila in dekadni sistem. Poglavja označujemo s števili od 1 dalje. Za številom ne pišemo pike. Iz velikosti in oblike črk naj bo razvidna hierarhija poglavij.

\medskip\noindent
V nadaljevanju je podan primer številčenja poglavij:

\bigskip\noindent
\begin{tabular}{lll}
poglavja 1, 2, 3 & \large VELIKE ČRKE, KREPKO & (npr. 14 pt) \\
poglavja 1.1, 1.2, 2.1 & \large VELIKE ČRKE & (npr. 14 pt) \\
poglavja 1.1.1, 1.1.2 & Male črke, krepko & (npr. 12 pt) \\
poglavja 1.1.1.1, 1.1.1.2 & Male črke, navadno & (npr. 12 pt)
\end{tabular}

\section{OBLIKOVANJE PREGLEDNIC, SLIK IN GRAFI\-KONOV}

Pri oblikovanju preglednic, slik in grafikonov upoštevajte, da:
\begin{itemize}
\item morajo imeti zaporedno številko in naslov;
\item naslov preglednice;
\item naslov slike ali grafikona.
\end{itemize}

Naslove priporočamo velikosti 10 pt, da se optično ločijo od besedila. Umeščenost naslova preglednice, slike ali grafikona se razlikuje glede na navodila za navajanje virov posameznega študijskega programa.

\section{ZAKLJUČNA NALOGA V ANGLEŠKEM JEZIKU: DALJŠI POVZETEK V SLOVENSKEM JEZIKU}

V kolikor je študentu v skladu s pravili fakultete odobrena priprava zaključne naloge v angleškem jeziku, mora študent pripraviti povzetek naloge v obsegu od 4.000 do 10.000 znakov (s presledki) v slovenskem jeziku. Povzetek naloge v slovenskem jeziku je zadnje poglavje naloge ter je ustrezno oštevilčeno (pred poglavjem Literatura in viri).

\section{LITERATURA IN VIRI}

Vsa literatura in viri uporabljeni v magistrskem delu morajo biti ustrezno citirani. Študenti kršijo postopek priprave zaključnega dela, če prepišejo besedila drugih avtorjev v celoti ali delno in pri tem ne citirajo avtorja (plagiatorstvo) oz. če delo ni rezultat študentovega lastnega dela.

Za sankcioniranje teh kršitev se uporabljajo določila Pravilnika o preverjanju in ocenjevanju znanja na Univerzi na Primorskem in Pravilnika o disciplinski odgovornosti študentov Univerze na Primorskem. Če pravila citiranja in priprave seznama literature niso določena na drugačen način na ravni posameznega študijskega programa (podrobnejša navodila na ravni posameznega oddelka fakultete), se pri citiranju in pripravi seznama literature uporabi spodnja določila.

Reference morajo biti citirane po abecednem redu prvega avtorja in v tekstu zapisane s številko v oglatem oklepaju, npr.: [1], [1, 2], [1, Theorem 1.5], [2, Corollary 9.4].